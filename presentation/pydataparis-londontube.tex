
% Default to the notebook output style

    


% Inherit from the specified cell style.




    
\documentclass{article}

    
    
    \usepackage{graphicx} % Used to insert images
    \usepackage{adjustbox} % Used to constrain images to a maximum size 
    \usepackage{color} % Allow colors to be defined
    \usepackage{enumerate} % Needed for markdown enumerations to work
    \usepackage{geometry} % Used to adjust the document margins
    \usepackage{amsmath} % Equations
    \usepackage{amssymb} % Equations
    \usepackage[mathletters]{ucs} % Extended unicode (utf-8) support
    \usepackage[utf8x]{inputenc} % Allow utf-8 characters in the tex document
    \usepackage{fancyvrb} % verbatim replacement that allows latex
    \usepackage{grffile} % extends the file name processing of package graphics 
                         % to support a larger range 
    % The hyperref package gives us a pdf with properly built
    % internal navigation ('pdf bookmarks' for the table of contents,
    % internal cross-reference links, web links for URLs, etc.)
    \usepackage{hyperref}
    \usepackage{longtable} % longtable support required by pandoc >1.10
    \usepackage{booktabs}  % table support for pandoc > 1.12.2
    

    
    
    \definecolor{orange}{cmyk}{0,0.4,0.8,0.2}
    \definecolor{darkorange}{rgb}{.71,0.21,0.01}
    \definecolor{darkgreen}{rgb}{.12,.54,.11}
    \definecolor{myteal}{rgb}{.26, .44, .56}
    \definecolor{gray}{gray}{0.45}
    \definecolor{lightgray}{gray}{.95}
    \definecolor{mediumgray}{gray}{.8}
    \definecolor{inputbackground}{rgb}{.95, .95, .85}
    \definecolor{outputbackground}{rgb}{.95, .95, .95}
    \definecolor{traceback}{rgb}{1, .95, .95}
    % ansi colors
    \definecolor{red}{rgb}{.6,0,0}
    \definecolor{green}{rgb}{0,.65,0}
    \definecolor{brown}{rgb}{0.6,0.6,0}
    \definecolor{blue}{rgb}{0,.145,.698}
    \definecolor{purple}{rgb}{.698,.145,.698}
    \definecolor{cyan}{rgb}{0,.698,.698}
    \definecolor{lightgray}{gray}{0.5}
    
    % bright ansi colors
    \definecolor{darkgray}{gray}{0.25}
    \definecolor{lightred}{rgb}{1.0,0.39,0.28}
    \definecolor{lightgreen}{rgb}{0.48,0.99,0.0}
    \definecolor{lightblue}{rgb}{0.53,0.81,0.92}
    \definecolor{lightpurple}{rgb}{0.87,0.63,0.87}
    \definecolor{lightcyan}{rgb}{0.5,1.0,0.83}
    
    % commands and environments needed by pandoc snippets
    % extracted from the output of `pandoc -s`
    \DefineVerbatimEnvironment{Highlighting}{Verbatim}{commandchars=\\\{\}}
    % Add ',fontsize=\small' for more characters per line
    \newenvironment{Shaded}{}{}
    \newcommand{\KeywordTok}[1]{\textcolor[rgb]{0.00,0.44,0.13}{\textbf{{#1}}}}
    \newcommand{\DataTypeTok}[1]{\textcolor[rgb]{0.56,0.13,0.00}{{#1}}}
    \newcommand{\DecValTok}[1]{\textcolor[rgb]{0.25,0.63,0.44}{{#1}}}
    \newcommand{\BaseNTok}[1]{\textcolor[rgb]{0.25,0.63,0.44}{{#1}}}
    \newcommand{\FloatTok}[1]{\textcolor[rgb]{0.25,0.63,0.44}{{#1}}}
    \newcommand{\CharTok}[1]{\textcolor[rgb]{0.25,0.44,0.63}{{#1}}}
    \newcommand{\StringTok}[1]{\textcolor[rgb]{0.25,0.44,0.63}{{#1}}}
    \newcommand{\CommentTok}[1]{\textcolor[rgb]{0.38,0.63,0.69}{\textit{{#1}}}}
    \newcommand{\OtherTok}[1]{\textcolor[rgb]{0.00,0.44,0.13}{{#1}}}
    \newcommand{\AlertTok}[1]{\textcolor[rgb]{1.00,0.00,0.00}{\textbf{{#1}}}}
    \newcommand{\FunctionTok}[1]{\textcolor[rgb]{0.02,0.16,0.49}{{#1}}}
    \newcommand{\RegionMarkerTok}[1]{{#1}}
    \newcommand{\ErrorTok}[1]{\textcolor[rgb]{1.00,0.00,0.00}{\textbf{{#1}}}}
    \newcommand{\NormalTok}[1]{{#1}}
    
    % Define a nice break command that doesn't care if a line doesn't already
    % exist.
    \def\br{\hspace*{\fill} \\* }
    % Math Jax compatability definitions
    \def\gt{>}
    \def\lt{<}
    % Document parameters
    \title{pydataparis-londontube}
    
    
    

    % Pygments definitions
    
\makeatletter
\def\PY@reset{\let\PY@it=\relax \let\PY@bf=\relax%
    \let\PY@ul=\relax \let\PY@tc=\relax%
    \let\PY@bc=\relax \let\PY@ff=\relax}
\def\PY@tok#1{\csname PY@tok@#1\endcsname}
\def\PY@toks#1+{\ifx\relax#1\empty\else%
    \PY@tok{#1}\expandafter\PY@toks\fi}
\def\PY@do#1{\PY@bc{\PY@tc{\PY@ul{%
    \PY@it{\PY@bf{\PY@ff{#1}}}}}}}
\def\PY#1#2{\PY@reset\PY@toks#1+\relax+\PY@do{#2}}

\expandafter\def\csname PY@tok@gd\endcsname{\def\PY@tc##1{\textcolor[rgb]{0.63,0.00,0.00}{##1}}}
\expandafter\def\csname PY@tok@gu\endcsname{\let\PY@bf=\textbf\def\PY@tc##1{\textcolor[rgb]{0.50,0.00,0.50}{##1}}}
\expandafter\def\csname PY@tok@gt\endcsname{\def\PY@tc##1{\textcolor[rgb]{0.00,0.27,0.87}{##1}}}
\expandafter\def\csname PY@tok@gs\endcsname{\let\PY@bf=\textbf}
\expandafter\def\csname PY@tok@gr\endcsname{\def\PY@tc##1{\textcolor[rgb]{1.00,0.00,0.00}{##1}}}
\expandafter\def\csname PY@tok@cm\endcsname{\let\PY@it=\textit\def\PY@tc##1{\textcolor[rgb]{0.25,0.50,0.50}{##1}}}
\expandafter\def\csname PY@tok@vg\endcsname{\def\PY@tc##1{\textcolor[rgb]{0.10,0.09,0.49}{##1}}}
\expandafter\def\csname PY@tok@m\endcsname{\def\PY@tc##1{\textcolor[rgb]{0.40,0.40,0.40}{##1}}}
\expandafter\def\csname PY@tok@mh\endcsname{\def\PY@tc##1{\textcolor[rgb]{0.40,0.40,0.40}{##1}}}
\expandafter\def\csname PY@tok@go\endcsname{\def\PY@tc##1{\textcolor[rgb]{0.53,0.53,0.53}{##1}}}
\expandafter\def\csname PY@tok@ge\endcsname{\let\PY@it=\textit}
\expandafter\def\csname PY@tok@vc\endcsname{\def\PY@tc##1{\textcolor[rgb]{0.10,0.09,0.49}{##1}}}
\expandafter\def\csname PY@tok@il\endcsname{\def\PY@tc##1{\textcolor[rgb]{0.40,0.40,0.40}{##1}}}
\expandafter\def\csname PY@tok@cs\endcsname{\let\PY@it=\textit\def\PY@tc##1{\textcolor[rgb]{0.25,0.50,0.50}{##1}}}
\expandafter\def\csname PY@tok@cp\endcsname{\def\PY@tc##1{\textcolor[rgb]{0.74,0.48,0.00}{##1}}}
\expandafter\def\csname PY@tok@gi\endcsname{\def\PY@tc##1{\textcolor[rgb]{0.00,0.63,0.00}{##1}}}
\expandafter\def\csname PY@tok@gh\endcsname{\let\PY@bf=\textbf\def\PY@tc##1{\textcolor[rgb]{0.00,0.00,0.50}{##1}}}
\expandafter\def\csname PY@tok@ni\endcsname{\let\PY@bf=\textbf\def\PY@tc##1{\textcolor[rgb]{0.60,0.60,0.60}{##1}}}
\expandafter\def\csname PY@tok@nl\endcsname{\def\PY@tc##1{\textcolor[rgb]{0.63,0.63,0.00}{##1}}}
\expandafter\def\csname PY@tok@nn\endcsname{\let\PY@bf=\textbf\def\PY@tc##1{\textcolor[rgb]{0.00,0.00,1.00}{##1}}}
\expandafter\def\csname PY@tok@no\endcsname{\def\PY@tc##1{\textcolor[rgb]{0.53,0.00,0.00}{##1}}}
\expandafter\def\csname PY@tok@na\endcsname{\def\PY@tc##1{\textcolor[rgb]{0.49,0.56,0.16}{##1}}}
\expandafter\def\csname PY@tok@nb\endcsname{\def\PY@tc##1{\textcolor[rgb]{0.00,0.50,0.00}{##1}}}
\expandafter\def\csname PY@tok@nc\endcsname{\let\PY@bf=\textbf\def\PY@tc##1{\textcolor[rgb]{0.00,0.00,1.00}{##1}}}
\expandafter\def\csname PY@tok@nd\endcsname{\def\PY@tc##1{\textcolor[rgb]{0.67,0.13,1.00}{##1}}}
\expandafter\def\csname PY@tok@ne\endcsname{\let\PY@bf=\textbf\def\PY@tc##1{\textcolor[rgb]{0.82,0.25,0.23}{##1}}}
\expandafter\def\csname PY@tok@nf\endcsname{\def\PY@tc##1{\textcolor[rgb]{0.00,0.00,1.00}{##1}}}
\expandafter\def\csname PY@tok@si\endcsname{\let\PY@bf=\textbf\def\PY@tc##1{\textcolor[rgb]{0.73,0.40,0.53}{##1}}}
\expandafter\def\csname PY@tok@s2\endcsname{\def\PY@tc##1{\textcolor[rgb]{0.73,0.13,0.13}{##1}}}
\expandafter\def\csname PY@tok@vi\endcsname{\def\PY@tc##1{\textcolor[rgb]{0.10,0.09,0.49}{##1}}}
\expandafter\def\csname PY@tok@nt\endcsname{\let\PY@bf=\textbf\def\PY@tc##1{\textcolor[rgb]{0.00,0.50,0.00}{##1}}}
\expandafter\def\csname PY@tok@nv\endcsname{\def\PY@tc##1{\textcolor[rgb]{0.10,0.09,0.49}{##1}}}
\expandafter\def\csname PY@tok@s1\endcsname{\def\PY@tc##1{\textcolor[rgb]{0.73,0.13,0.13}{##1}}}
\expandafter\def\csname PY@tok@kd\endcsname{\let\PY@bf=\textbf\def\PY@tc##1{\textcolor[rgb]{0.00,0.50,0.00}{##1}}}
\expandafter\def\csname PY@tok@sh\endcsname{\def\PY@tc##1{\textcolor[rgb]{0.73,0.13,0.13}{##1}}}
\expandafter\def\csname PY@tok@sc\endcsname{\def\PY@tc##1{\textcolor[rgb]{0.73,0.13,0.13}{##1}}}
\expandafter\def\csname PY@tok@sx\endcsname{\def\PY@tc##1{\textcolor[rgb]{0.00,0.50,0.00}{##1}}}
\expandafter\def\csname PY@tok@bp\endcsname{\def\PY@tc##1{\textcolor[rgb]{0.00,0.50,0.00}{##1}}}
\expandafter\def\csname PY@tok@c1\endcsname{\let\PY@it=\textit\def\PY@tc##1{\textcolor[rgb]{0.25,0.50,0.50}{##1}}}
\expandafter\def\csname PY@tok@kc\endcsname{\let\PY@bf=\textbf\def\PY@tc##1{\textcolor[rgb]{0.00,0.50,0.00}{##1}}}
\expandafter\def\csname PY@tok@c\endcsname{\let\PY@it=\textit\def\PY@tc##1{\textcolor[rgb]{0.25,0.50,0.50}{##1}}}
\expandafter\def\csname PY@tok@mf\endcsname{\def\PY@tc##1{\textcolor[rgb]{0.40,0.40,0.40}{##1}}}
\expandafter\def\csname PY@tok@err\endcsname{\def\PY@bc##1{\setlength{\fboxsep}{0pt}\fcolorbox[rgb]{1.00,0.00,0.00}{1,1,1}{\strut ##1}}}
\expandafter\def\csname PY@tok@mb\endcsname{\def\PY@tc##1{\textcolor[rgb]{0.40,0.40,0.40}{##1}}}
\expandafter\def\csname PY@tok@ss\endcsname{\def\PY@tc##1{\textcolor[rgb]{0.10,0.09,0.49}{##1}}}
\expandafter\def\csname PY@tok@sr\endcsname{\def\PY@tc##1{\textcolor[rgb]{0.73,0.40,0.53}{##1}}}
\expandafter\def\csname PY@tok@mo\endcsname{\def\PY@tc##1{\textcolor[rgb]{0.40,0.40,0.40}{##1}}}
\expandafter\def\csname PY@tok@kn\endcsname{\let\PY@bf=\textbf\def\PY@tc##1{\textcolor[rgb]{0.00,0.50,0.00}{##1}}}
\expandafter\def\csname PY@tok@mi\endcsname{\def\PY@tc##1{\textcolor[rgb]{0.40,0.40,0.40}{##1}}}
\expandafter\def\csname PY@tok@gp\endcsname{\let\PY@bf=\textbf\def\PY@tc##1{\textcolor[rgb]{0.00,0.00,0.50}{##1}}}
\expandafter\def\csname PY@tok@o\endcsname{\def\PY@tc##1{\textcolor[rgb]{0.40,0.40,0.40}{##1}}}
\expandafter\def\csname PY@tok@kr\endcsname{\let\PY@bf=\textbf\def\PY@tc##1{\textcolor[rgb]{0.00,0.50,0.00}{##1}}}
\expandafter\def\csname PY@tok@s\endcsname{\def\PY@tc##1{\textcolor[rgb]{0.73,0.13,0.13}{##1}}}
\expandafter\def\csname PY@tok@kp\endcsname{\def\PY@tc##1{\textcolor[rgb]{0.00,0.50,0.00}{##1}}}
\expandafter\def\csname PY@tok@w\endcsname{\def\PY@tc##1{\textcolor[rgb]{0.73,0.73,0.73}{##1}}}
\expandafter\def\csname PY@tok@kt\endcsname{\def\PY@tc##1{\textcolor[rgb]{0.69,0.00,0.25}{##1}}}
\expandafter\def\csname PY@tok@ow\endcsname{\let\PY@bf=\textbf\def\PY@tc##1{\textcolor[rgb]{0.67,0.13,1.00}{##1}}}
\expandafter\def\csname PY@tok@sb\endcsname{\def\PY@tc##1{\textcolor[rgb]{0.73,0.13,0.13}{##1}}}
\expandafter\def\csname PY@tok@k\endcsname{\let\PY@bf=\textbf\def\PY@tc##1{\textcolor[rgb]{0.00,0.50,0.00}{##1}}}
\expandafter\def\csname PY@tok@se\endcsname{\let\PY@bf=\textbf\def\PY@tc##1{\textcolor[rgb]{0.73,0.40,0.13}{##1}}}
\expandafter\def\csname PY@tok@sd\endcsname{\let\PY@it=\textit\def\PY@tc##1{\textcolor[rgb]{0.73,0.13,0.13}{##1}}}

\def\PYZbs{\char`\\}
\def\PYZus{\char`\_}
\def\PYZob{\char`\{}
\def\PYZcb{\char`\}}
\def\PYZca{\char`\^}
\def\PYZam{\char`\&}
\def\PYZlt{\char`\<}
\def\PYZgt{\char`\>}
\def\PYZsh{\char`\#}
\def\PYZpc{\char`\%}
\def\PYZdl{\char`\$}
\def\PYZhy{\char`\-}
\def\PYZsq{\char`\'}
\def\PYZdq{\char`\"}
\def\PYZti{\char`\~}
% for compatibility with earlier versions
\def\PYZat{@}
\def\PYZlb{[}
\def\PYZrb{]}
\makeatother


    % Exact colors from NB
    \definecolor{incolor}{rgb}{0.0, 0.0, 0.5}
    \definecolor{outcolor}{rgb}{0.545, 0.0, 0.0}



    
    % Prevent overflowing lines due to hard-to-break entities
    \sloppy 
    % Setup hyperref package
    \hypersetup{
      breaklinks=true,  % so long urls are correctly broken across lines
      colorlinks=true,
      urlcolor=blue,
      linkcolor=darkorange,
      citecolor=darkgreen,
      }
    % Slightly bigger margins than the latex defaults
    
    \geometry{verbose,tmargin=1in,bmargin=1in,lmargin=1in,rmargin=1in}
    
    

    \begin{document}
    
    
    \maketitle
    
    

    
    \section{Rush Hour Dynamics: Using Python to Study the London
Underground}\label{rush-hour-dynamics-using-python-to-study-the-london-underground}

\subsubsection{Camilla Montonen}\label{camilla-montonen}

\subsubsection{PyData Paris 2015}\label{pydata-paris-2015}

    \section{Introduction}\label{introduction}

    \section{Roadmap}\label{roadmap}

\begin{enumerate}
\def\labelenumi{\arabic{enumi}.}
\itemsep1pt\parskip0pt\parsep0pt
\item
  Motivation: Why would you want to analyse the London Underground?!
  Commuting on it is bad enough.
\item
  Data collection: The Challenge of Collecting Data Stored in a Map
\item
  Data analysis: Leveraging graph-tool to analyse the London Underground
\item
  Simulations: Creating simulations using Bokeh
\end{enumerate}

    \section{The Takeaway Message}\label{the-takeaway-message}

There are interesting data problems everywhere in our environment and
Python provides a set of amazing tools to start asking questions and
generating answers. Don't be afraid to investigate (and write some
amazing Python code as you go)!

    \section{Back in August 2014\ldots{}}\label{back-in-august-2014}

    \section{Which Tube line should I take to
work?}\label{which-tube-line-should-i-take-to-work}

    \subsection{Motivations}\label{motivations}

\begin{itemize}
\itemsep1pt\parskip0pt\parsep0pt
\item
  Delays and suspension on remote stations and Tube lines seem to
  congest even remote stations
\end{itemize}

Studying the London Underground and performing simulations can help to
understand:

\begin{itemize}
\itemsep1pt\parskip0pt\parsep0pt
\item
  the overall characteristics of the London Underground Network
\item
  The Tube stations that are critical for the proper functioning of the
  network
\end{itemize}

    \section{Data Collection}\label{data-collection}

The challenge in data collection was to find a way to translate the
standard TfL London Underground Map into a graph representation.

Start:

Goal:

    \subsection{Data collection:}\label{data-collection}

It would be cool to program some kind of OCR to automatically read the
data from the map and produce a data file! But alas, I had to resort to
manually creating a data file:

\begin{verbatim}
#Station #Neighbour(line)
Acton Town          Chiswick Park (District), South Ealing (Picadilly), Turnham Green (Picadilly)
Aldgate             Tower Hill (Circle; District), Liverpool Street (Metropolitan; Circle; District)
Aldgate East        Tower Hill (District), Liverpool Street (HammersmithCity; Metropolitan)
Alperton            Sudbury Town (Picadilly), Park Royal (Picadilly)
\end{verbatim}

    \section{Data analysis}\label{data-analysis}

Now that we have collected out data, it's time start interesting
questions about it.

Some of the burning questions that I had:

\begin{enumerate}
\def\labelenumi{\arabic{enumi}.}
\itemsep1pt\parskip0pt\parsep0pt
\item
  What are the most ``important'' stations in the London Underground
  network?
\item
  What is the average shortest path between any two stations?
\item
  Which stations are the most critical for the proper functioning of the
  network?
\end{enumerate}

    \section{Data analysis using
graph-tool}\label{data-analysis-using-graph-tool}

\begin{itemize}
\item
  \texttt{graph-tool} is a Python library written by Tiago Peixoto that
  provides a number of tools for analyzing and plotting graphs.
\item
  it provides a number of useful tools and methods

  \begin{enumerate}
  \def\labelenumi{\arabic{enumi}.}
  \itemsep1pt\parskip0pt\parsep0pt
  \item
    A \texttt{Graph} object for defining graphs
  \item
    Property maps : helpful for associating values with vertices and
    edges
  \item
    Various methods for analyzing graph topology
  \item
    Built-in graph visualization
  \end{enumerate}
\end{itemize}

Another Python tool that you may wish to explore for graph analysis is
\texttt{NetworkX}.

    \section{Data analysis using
graph-tool}\label{data-analysis-using-graph-tool}

Let's see what the graph looks like!

    \section{Data analysis using
graph-tool}\label{data-analysis-using-graph-tool}

\texttt{graph-tool} allows us to compute several interesting metrics,
which are often used to characterize graphs:

\begin{enumerate}
\def\labelenumi{\arabic{enumi}.}
\itemsep1pt\parskip0pt\parsep0pt
\item
  Degree distribution
\item
  Average shortest path
\end{enumerate}

    \begin{Verbatim}[commandchars=\\\{\}]
{\color{incolor}In [{\color{incolor}2}]:} \PY{c}{\PYZsh{} import necessary packages}
        
        \PY{c}{\PYZsh{} define data files}
        \PY{n}{geographical\PYZus{}data}\PY{o}{=}\PY{l+s}{\PYZdq{}}\PY{l+s}{/home/winterflower/programming\PYZus{}projects/python\PYZhy{}londontube/src/data/london\PYZus{}stations.csv}\PY{l+s}{\PYZdq{}}
        \PY{n}{network\PYZus{}data}\PY{o}{=}\PY{l+s}{\PYZdq{}}\PY{l+s}{/home/winterflower/programming\PYZus{}projects/python\PYZhy{}londontube/src/data/londontubes.txt}\PY{l+s}{\PYZdq{}}
\end{Verbatim}

    \section{Data analysis using
graph-tool}\label{data-analysis-using-graph-tool}

\begin{enumerate}
\def\labelenumi{\arabic{enumi}.}
\itemsep1pt\parskip0pt\parsep0pt
\item
  What are the most ``important'' stations in the London Underground
  network?
\end{enumerate}

Of course, there are many ways to measure the importance of a vertex in
a graph. One such measure is called \emph{betweenness centrality} .
Simply stated, it measures the fraction of shortest paths out of all
shortest paths that pass through the vertex.

\texttt{graph-tool} provides a module \texttt{graph-tool.centrality}
which allows you to compute various centrality measures out of the box.

    \subsection{Betweenness Centrality}\label{betweenness-centrality}

What fraction of all shortest paths passes through this vertex?

    \begin{Verbatim}[commandchars=\\\{\}]
{\color{incolor}In [{\color{incolor}2}]:} \PY{c}{\PYZsh{}define some useful preliminaries}
        \PY{n}{geographical\PYZus{}data}\PY{o}{=}\PY{l+s}{\PYZdq{}}\PY{l+s}{/home/winterflower/programming\PYZus{}projects/python\PYZhy{}londontube/src/data/london\PYZus{}stations.csv}\PY{l+s}{\PYZdq{}}
        \PY{n}{network\PYZus{}data}\PY{o}{=}\PY{l+s}{\PYZdq{}}\PY{l+s}{/home/winterflower/programming\PYZus{}projects/python\PYZhy{}londontube/src/data/londontubes.txt}\PY{l+s}{\PYZdq{}}
        
        \PY{c}{\PYZsh{}calculate the betweenness centrality}
        \PY{c}{\PYZsh{}create the map\PYZus{}object}
        
        \PY{k+kn}{from} \PY{n+nn}{src} \PY{k+kn}{import} \PY{n}{simulation\PYZus{}utils}
        \PY{k+kn}{from} \PY{n+nn}{src.graph\PYZus{}analytics} \PY{k+kn}{import} \PY{n}{graph\PYZus{}analysis}
        \PY{k+kn}{import} \PY{n+nn}{pandas} \PY{k+kn}{as} \PY{n+nn}{pd}
        \PY{n}{betweenness\PYZus{}centrality\PYZus{}series\PYZus{}object}\PY{o}{=}\PY{n}{graph\PYZus{}analysis}\PY{o}{.}\PY{n}{calculate\PYZus{}betweenness}\PY{p}{(}\PY{n}{network\PYZus{}data}\PY{p}{)}
        \PY{n}{betweenness\PYZus{}centrality\PYZus{}series\PYZus{}object}\PY{o}{.}\PY{n}{sort}\PY{p}{(}\PY{n}{ascending}\PY{o}{=}\PY{n+nb+bp}{False}\PY{p}{)}
        \PY{k}{print} \PY{n}{betweenness\PYZus{}centrality\PYZus{}series\PYZus{}object}\PY{p}{[}\PY{p}{:}\PY{l+m+mi}{10}\PY{p}{]}
\end{Verbatim}

    \begin{Verbatim}[commandchars=\\\{\}]
Baker Street               0.344084
King's Cross St.Pancras    0.303868
Liverpool Street           0.267392
Green Park                 0.263264
Mile End                   0.229449
Bethnal Green              0.227822
Victoria                   0.222771
Stratford                  0.220119
Finchley Road              0.211660
Waterloo                   0.207129
dtype: float64
    \end{Verbatim}

    \subsection{Betweenness Centrality}\label{betweenness-centrality}

    \subsection{Betweenness Centrality}\label{betweenness-centrality}

    \subsection{Shortest paths}\label{shortest-paths}

Which station has the smallest average shortest path to any other
station in the graph?

    \begin{Verbatim}[commandchars=\\\{\}]
{\color{incolor}In [{\color{incolor}5}]:} \PY{c}{\PYZsh{}\PYZsh{} calculate the length of the shortest path from any two stations}
        
        \PY{k+kn}{from} \PY{n+nn}{src.graph\PYZus{}analytics} \PY{k+kn}{import} \PY{n}{graph\PYZus{}analysis}
        \PY{n}{shortest\PYZus{}paths}\PY{o}{=}\PY{n}{graph\PYZus{}analysis}\PY{o}{.}\PY{n}{calculate\PYZus{}all\PYZus{}shortest\PYZus{}paths}\PY{p}{(}\PY{n}{network\PYZus{}data}\PY{p}{)}
        \PY{c}{\PYZsh{}calculate the mean shortest path}
        \PY{n}{mean\PYZus{}shortest\PYZus{}path}\PY{o}{=}\PY{n}{shortest\PYZus{}paths}\PY{o}{.}\PY{n}{mean}\PY{p}{(}\PY{n}{axis}\PY{o}{=}\PY{l+m+mi}{0}\PY{p}{)}
        \PY{c}{\PYZsh{}find out stations with smallest mean shortest paths}
        \PY{n}{mean\PYZus{}shortest\PYZus{}path}\PY{o}{.}\PY{n}{order}\PY{p}{(}\PY{n}{ascending}\PY{o}{=}\PY{n+nb+bp}{True}\PY{p}{,} \PY{n}{inplace}\PY{o}{=}\PY{n+nb+bp}{True}\PY{p}{)}
        \PY{c}{\PYZsh{}find out the top 5 stations}
        \PY{n}{mean\PYZus{}shortest\PYZus{}path}\PY{p}{[}\PY{p}{:}\PY{l+m+mi}{5}\PY{p}{]}
\end{Verbatim}

            \begin{Verbatim}[commandchars=\\\{\}]
{\color{outcolor}Out[{\color{outcolor}5}]:} Green Park       8.901887
        Oxford Circus    9.007547
        Bond Street      9.090566
        Baker Street     9.211321
        Westminster      9.339623
        dtype: float64
\end{Verbatim}
        
    \section{Simulating commuter flow between
stations}\label{simulating-commuter-flow-between-stations}

Designing a simple two component simulation:

    \subsection{Simulating commuter flow between
stations}\label{simulating-commuter-flow-between-stations}

\begin{itemize}
\itemsep1pt\parskip0pt\parsep0pt
\item
  Bokeh allows you to create graphs that update in ``real-time''
\end{itemize}

    \subsection{Summary}\label{summary}

\begin{itemize}
\itemsep1pt\parskip0pt\parsep0pt
\item
  Python provides excellent libraries for studying real-world problems
  where the natural representation of the data is a graph
\item
  In addition to calculating metrics, you can easily make amazing
  animations by integrating graph-tool with bokeh
\item
  Find interesting problems, ask hard questions and start exploring!
\end{itemize}

    \subsection{Thank you ! (and please ask
questions!)}\label{thank-you-and-please-ask-questions}

    \begin{Verbatim}[commandchars=\\\{\}]
{\color{incolor}In [{\color{incolor}}]:} 
\end{Verbatim}


    % Add a bibliography block to the postdoc
    
    
    
    \end{document}
